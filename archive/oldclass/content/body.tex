% Here is where you put your thesis. :-)  Something like:
% 
% \chapter{Introduction}
%
% some text blah blah blah
%
% \chapter{Prior work}
%
% More text blah blah
%
%
% If you want chapters to begin on new pages, I recommend 
% manually putting a \newpage before each chapter.  It's a
% pain to put this in the .cls file because the title goes
% on the same page as the introduction (note that chapter
% one does NOT have a \newpage before it)
%


\chapter{Introduction}

Lemurs come equipped with large, almost shining eyes which
are legendary for reflecting the light of campfires back at people
gathered in clearings with curious Lemurs in the nearby trees.
Lemurs are said to have "googly" eyes which are used in communica-
tion -- e.g. winking, rolling, staring, etc. in addition to spoken
words.  When a Lemur is around a good-looking Lemur of the oppo-
site sex, you can generally tell that the first Lemur finds the
second one attractive as the first Lemur will "get all googly and
stuff" (in the words of Chris Karluk).

\chapter{What do Lemurs do when the weather turns cold?}

Word has it that Lemurs travel via subways and steam tunnels when
the weather turns cold, "moving in" with friendly humans, thereby
assured of a warm dwelling place and lots of Big K grape soda for
the duration of the winter.

Apparently the humans they move in with are persuaded to share
their living quarters with some Lemurs if the Lemurs let their
"hosts"  use their blaster pistols now and then.  "Negotiations"
with the invading Lemurs can be interesting, as this exchange of
messages in a recent case shows:

        The humans said that the Lemurs could come by if they
         promised not to annoy the neighbors, play the stereo too
         loud, hog the computers and modem, use laserdisks as
         frisbees, or swing from the kitchen light.

        The Lemurs countered by asking if they could jump up and
         down on the bed.  Their representative said that he 
         couldn't guarantee all the above, but that they MUST be
         allowed to jump on the bed, or else they would come over
         anyway and do anything they want.  Furthermore, swinging
         from the kitchen light is something that all Lemurs
         instinctively love doing.   Would the humans object if they
         brought their own kitchen lights to attach to the kitchen
         ceiling?

        Sighing, the humans assented provided that the Lemurs
         either promise to repaint the kitchen walls afterwards or
         wear flip-flops.

        The Lemurs agreed and moved in.

\chapter{What do Lemurs like to eat?}

Legend has it that Lemurs love junk food.  Specifically Hostess
Twinkies, but also such things as generic snack cakes, cookies,
deviled eggs, pigs-in-a-blanket, squirt cheese on crackers, etc.
In other words, your average Lemur would be very content raiding
the hors d'oeuvres line at a cheap wedding reception.  Rumor has
it that Lemurs occasionally fall victim to strange cravings, such
as chocolate cakes with cherry pie filling and whipped cream on
top... and sauerkraut!  Sauerkraut on everything!!!  Let's not
explore this subject any further.

Lemurs like to eat.  This they do well: it's not uncommon for a
Lemur to devour the entire contents of a candy machine in under
ten hours.  (Lemurs often can squirm inside the machine via the
slots at the bottom, eat their fill, and then have trouble getting
back out.  vending machine repairmen often find engorged Lemurs
sitting in a pile of Mounds bar wrappers looking woeful.  The
Lemurs are usually deported back to the Duke University primate
home.)  Lemurs are nothing if not pragmatic.  A vending machine
full of Mars bars down the hall from the office they've taken over
is greatly preferred to one a few buildings away that contains
Twinkies.  Besides, those machines are usually long since cleaned
out by those few Lemurs who do forage afar.

\chapter{What is the "Lemurcon equation"?}

D. Harmon writes:

You may have heard of a type of graph called a limacon, which is a
graphed from the function r=a+b*cosh or r=a+b*sinh.  What you
probably haven't heard of is another similar type of function
called a Lemurcon.  This function is the equation l=e(mu*r),
where l is the length of the radius, r is a constant which has
several different values for each value of mu, and mu is the
independent variable.

Since the discovery of Lemurcon equation is fairly recent, there
is an opportunity to immortalize your name in history by discover-
ing other graphs of the Lemurcon equation.  
