\headingpage{Using \beavtex }{John W. Metta, Leslie A. Aamport and Donald K. Nuth}{Journal of \LaTeX{}\ Fun}{Oregon State University\\Corvallis, Oregon}{Vol. 4, Num 5, pp. 234-345}


\chapter{Using \beavtex}

The first thing you should notice about \beavtex is this documentation.
It is formatted in the manner of a dual masters thesis. Even the previous
page is a cover page for those who choose manuscript style and have
published a chapter before presenting their thesis. Most (it should
be all, but I may have missed some) formatting requirements of the
Oregon State thesis guide have been taken care of, and the design
of this documentation is such that many optional requirements are
used so that you have an example.


\section{Loading the class}

The basic options of \beavtex are similiar to those in any class,
and \beavtex itself is based on the book class. Thus, you can load
the class with

\begin{lyxcode}
\textbackslash{}documentclass{[}opts{]}\{beavtex\}
\end{lyxcode}
and any options that you pass as \texttt{opts} and that \beavtex
does not recognize, will go straight on to the book class. Thus, if
you want to format your thesis to print on both sides of the page,
you would load the class as

\begin{lyxcode}
\textbackslash{}documentclass{[}twoside{]}\{beavtex\}
\end{lyxcode}
Since \beavtex does not recognize the \texttt{twoside} option, it
will pass to the underlying book class. 

\begin{lyxlist}{00.00.0000}
\item [Margins]The graduate school requires margins that are 1.5 inch left
and 1 inch elsewhere \emph{at a minimum}. They recommend 1.7 inch
left. \beavtex natively sets a 1.7 inch left margin, which you can
change to 1.5 inch by using the option \texttt{\noun{1.5}}. Regardless
of your choice of inner margin, \beavtex will set your outer margins
at 1.1 inch, to assure that your document conforms to the graduate
school requirements.
\item [Font~Size]You may use the options \texttt{12pt}, \noun{}\texttt{11pt},
\noun{}and \noun{}\texttt{10pt} to set the base font size. The
default is a 12 point font. 
\end{lyxlist}
As an example, to print on both sides with a 1.5 inch inner margin
and an 11 point font, use:

\begin{lyxcode}
\textbackslash{}documentclass{[}twoside,1.5,11pt{]}\{beavtex\}
\end{lyxcode}
and the class will load, passing the \texttt{twoside} option to the
book class.


\section{Basic Commands}

Since \beavtex was designed to take care of all the pretext page
formatting automagically, you should only need to give it the basic
information. All of the following commands should be written as:

\begin{lyxcode}
\textbackslash{}command\{This~is~what~you~want~printed\}
\end{lyxcode}
Most commands have a \LyX{} counterpart in the master layout file
(the chapter layout file holds no \beavtex commands) and are noted
below.

\emph{Commands with no \LyX{} counterpart can still be used in \LyX{}
easily by using} \textsf{\emph{Layout\lyxarrow{}Document}} \emph{and
adding them to the preamble.}


\subsection{Mandatory Commands}

All mandatory commands have \LyX{} counterparts.

\begin{lyxlist}{00.00.0000}
\item [\texttt{\textbackslash{}title}]This is the title of your thesis.
If your title is very long, you will be interested in the optional
command \texttt{\textbackslash{}twotitle}, given below. 
\item [\texttt{\textbackslash{}author}]In short, you. Do not use the \LaTeX{}
\texttt{\textbf{\noun{\textbackslash{}and}}} command. Ideally, your
thesis should be one author only. If you must use more than one author,
you can simply type something like:

\begin{lyxcode}
\textbackslash{}author\{U.R.~Wise~and~I.M.~Dumb\}
\end{lyxcode}
\item [\texttt{\textbackslash{}major}]The official name of your major,
such as Geography, or Bioresource Engineering. Two majors are allowed
(See Sections~\ref{sub:Optional-Commands} and~\ref{sec:Advanced-Options}).
\item [\texttt{\textbackslash{}department}]The name of your department,
such as Geosciences or Bioengineering. Two departments are allowed
(See Sections~\ref{sub:Optional-Commands} and~\ref{sec:Advanced-Options}).
\item [\texttt{\textbackslash{}advisor}]The full name (including middle
initial) of your advisor. Two co-advisors are allowed (See Sections~\ref{sub:Optional-Commands}
and~\ref{sec:Advanced-Options}).
\item [\texttt{\textbackslash{}degree}]This will be something like {}``Masters
of Science'' or {}``Doctor of Philosophy''.
\item [\texttt{\textbackslash{}doctype}]The document type is probably either
{}``Thesis'' or {}``Dissertation,'' though you may define your
own, such as {}``Project'' or {}``Presentation''. \emph{(\LyX{}:
Document Type)}.
\end{lyxlist}

\subsubsection{Default Commands}

These commands are either optional, or are mandatory but have default
values built in. Not all have \LyX{} counterparts.

\begin{lyxlist}{00.00.0000}
\item [\texttt{\textbackslash{}abstract}]This should be the \emph{text}
of the abstract only--- the formal stuff is preformatted. The abstract
is formatted on the first page of the thesis. If for some reason,
you do not need an abstract, the abstract will still be formatted,
but you needn't print the first page. This page does not get numbered,
so will not affect numbering of subsequent pages if you discard it.
\emph{(\LyX{})}
\item [\texttt{\textbackslash{}submitdate}]The submission date of your
thesis. This value defaults to \texttt{\textbackslash{}today} if it
is not present, but you can enter whatever day you wish. This is the
date printed on the abstract and signature pages. \emph{(\LyX{}: Submission
Date)}
\item [\texttt{\textbackslash{}copyrightyear}]This is the year of your
document's copyright. It defaults to the current year (\texttt{\textbackslash{}the\textbackslash{}year}),
but can be changed. \emph{(\LyX{})}
\item [\texttt{\textbackslash{}commencementyear}]The year of your commencement.
Default is the current year. \emph{(\LyX{})}
\item [\texttt{\textbackslash{}depthead}]This is the title of the head
of your department, school, whatever. It defaults to {}``Chair,''
but you can use {}``Head,'' {}``Dean'' or whatever you want. \emph{(No
\LyX{})}
\item [\texttt{\textbackslash{}depttype}]This is the type of the department.
It defaults to {}``Department,'' but {}``School,'' {}``College,''
etc. are sometimes necessary. \emph{(No \LyX{})}
\end{lyxlist}

\subsection{Optional Commands\label{sub:Optional-Commands}}

Some optional reqirements, including some for those who are dual majors.
Many of these options were used to format this documentation, so that
you can see how they look.

\begin{lyxlist}{00.00.0000}
\item [\texttt{\textbackslash{}twotitle}]If your title is long, it will
overrun the margin on the abstract page. This is a serious \TeX{}
problem, that is even mentioned in the \TeX{}book as being difficult
at best to solve~\citep{knuth.84:TeXbook}. For this reason, you
should format your title completely with the \texttt{\noun{\textbackslash{}}}\texttt{title}
command, then view the dvi or pdf file to see where your title should
have a line break. You can then put the portion of your title after
the point of a necessary linebreak into this command, and it will
look fine. This does not affect the titlepage or anything else, so
if you are not formatting an abstract, you needn't worry about using
\texttt{\textbackslash{}twotitle}. (\emph{\LyX{}: Second Title)}
\item [\texttt{\textbackslash{}coadvisor}]Your co-advisor, or your second
advisor if you are a dual/double major. (\emph{\LyX{})}
\item [\texttt{\textbackslash{}twomajor}]This is your second major if you
are a dual/double major. (\emph{\LyX{}: Second Major)}
\item [\texttt{\textbackslash{}twodepartment}]Likewise for your department.
(\emph{\LyX{}: Second Department)}
\item [\texttt{\textbackslash{}twodepthead}]The title of the head of your
second department. (\emph{No \LyX{})}
\item [\texttt{\textbackslash{}twodepttype}]The type of your second department.
(\emph{No \LyX{})}
\end{lyxlist}

\subsubsection{Optional pages}

The following pages only print if you use the commands.

\begin{lyxlist}{00.00.0000}
\item [\texttt{\textbackslash{}acknowledgements}]These are your acknowledgements
usually in paragraph form, but formatted however you like. (\emph{\LyX{})}
\item [\texttt{\textbackslash{}preface}]Your preface. (\emph{\LyX{})}
\item [\texttt{\textbackslash{}dedication}]Your dedications page. (\emph{\LyX{})}
\item [\texttt{\textbackslash{}contributors}]For manuscript format, these
are the contributions of other workers/authors. (\emph{\LyX{})}
\end{lyxlist}

\section{Advanced Options\label{sec:Advanced-Options}}

There are many small things that can be different from project to
project, and thus much that is unique to each thesis/dissertation.
I've tried to capture all that I found here. Those that are not here,
you may have to add yourself or contact me and I can probably add
them. \emph{None of these commands have \LyX{} counterparts, all of
them must go into the preamble.}

\begin{lyxlist}{00.00.0000}
\item [\texttt{\textbackslash{}chapword}]This is the name (not the title)
of all chapters. It defaults to blank, which will just print a chapter
number. For an example, this document uses the command

\begin{lyxcode}
\textbackslash{}chapword\{\textbackslash{}beavtex~Chapter\}
\end{lyxcode}
so that you can see what it looks like to change the word.

\item [\texttt{\textbackslash{}chapheadsep}]This is a field which separates
the chapter number and the chapter title. It defaults to a space and
a double hyphen as in {}``Chapter 1 -- Some Title,'' but you can
choose to make this a single period with

\begin{lyxcode}
\textbackslash{}chapheadsep\{.\}
\end{lyxcode}
which would read {}``Chapter 1. Some Title.'' unless \texttt{\textbackslash{}chapword}
is not defined, in which case it would read {}``1. Some Title.''

\item [\texttt{\textbackslash{}nopretext}]Often, during the writing of
your thesis, you may want to check out how the formatting looks or
proof-read it as a document. For this, it may be cumbersome to print
all of the pretext pages only to ignore them and have to scroll past
them to get to your thesis. using the option

\begin{lyxcode}
\textbackslash{}nopretext\{\}
\end{lyxcode}
will force \LaTeX{} to ignore all of the pages that come before your
table of contents.

\item [\texttt{\textbackslash{}depthoftoc}]This controls how many levels
of sections and subsections are listed in the Table of Contents. The
number defines levels \emph{below} the chapter level. It defaults
to 2, meaning that sections and subsections will be listed in the
TOC. If you want, for instance, subsubsection and paragraph headings,
you would use

\begin{lyxcode}
\textbackslash{}depthoftoc\{4\}
\end{lyxcode}
which may be quite busy and awkward, but it's your thesis, not mine.
This also comes in handy in context with the \texttt{\textbackslash{}nopretext\{\}}
option, since you can set the depth to zero and only print chapter
names, ensuring that you don't need to scroll through a long TOC to
read your text. You can also comment out the \texttt{\textbackslash{}tableofcontents}
command to avoid printing it at all.

\item [\texttt{\textbackslash{}smallitem}]An itemize environment that is
used exactly like the \LaTeX{} \texttt{\textbackslash{}itemize}, but
which has smaller spaces between the text.

\begin{itemize}
\item This is a list with the normal \texttt{\textbackslash{}itemize} environment.
\item It has rather large spacing between the items.
\end{itemize}
versus the \texttt{\textbackslash{}smallitem} environment

\begin{smallitem}
\item The \texttt{\textbackslash{}smallitem} environment
\item has smaller spaces between each item.
\end{smallitem}
\item [\texttt{\textbackslash{}smallenum}]This is an enumerate environtment
that is used exactly like the \LaTeX{} \texttt{\textbackslash{}enumerate},
but which has spacing like \texttt{\textbackslash{}smallitem}.

\begin{smallenum}
\item Like the \texttt{\textbackslash{}smallitem} environtment,
\item the \texttt{\textbackslash{}smallenum} environment
\item is similiarly spaced.
\end{smallenum}
\item [\texttt{\textbackslash{}headingpage}]For those using the manuscript
layout, there is the possibility that you have published a chapter
of your thesis prior to submitting it to the graduate school. These
chapters require a cover page (as in the beginning of this chapter)
stating the title of paper (this must match the chapter title exactly!),
the authors, and the journal information. This page can be automagically
formatted with a single command:

\begin{lyxcode}
\textbackslash{}headingpage\{Title\}\{Authors\}\{Journal~Title\}\{Journal~address\}~\\
\{Issue~Information\}
\end{lyxcode}
where any of the fields can include a newline in the form of a \LaTeX{}
double slash (\texttt{\textbackslash{}\textbackslash{}}). This command
must come before the chapter title in the document. For \LyX{} users:
Use the \TeX{} entry method, which is the standard method that \TeX{}/\LaTeX{}
commands are entered. Again, do not use \texttt{\textbackslash{}and}
in the author list. 

\item [\texttt{\textbackslash{}bigfloatskip}]This is an option for setting
a wider separation for floats. It defaults to false, but you may use

\begin{lyxcode}
\textbackslash{}bigfloatskip\{true\}
\end{lyxcode}
to have 42pt, vs. 20pt spacing.

\item [\texttt{\textbackslash{}begin\{figure\}}]Figures are placed in your
document with

\begin{lyxcode}
\textbackslash{}begin\{figure\}\textbackslash{}end\{figure\}~
\end{lyxcode}
This will also allow for a separate list of figures to be included
in your Table of Context. If you have enough figures that you require
them to be in an appendix, use

\begin{lyxcode}
\textbackslash{}begin\{afigure\}\textbackslash{}end\{afigure\}
\end{lyxcode}
\item [\texttt{\textbackslash{}listoffigures}]This works exactly as it
should. It puts a List of Figures, formatted just as the Table of
Contents. To use it, place

\begin{lyxcode}
\textbackslash{}listoffigures\textbackslash{}clearpage
\end{lyxcode}
immediately after the TOC. You can also use

\begin{lyxcode}
\textbackslash{}listofappendixfigures\textbackslash{}clearpage
\end{lyxcode}
to format a list of figures that have been inserted with \texttt{\textbackslash{}begin\{afigure\}}.

\item [\texttt{\textbackslash{}begin\{table\}}]This works the same as \texttt{\textbackslash{}begin\{figure\}}
above. Use \texttt{\textbackslash{}begin\{atable\}} for appendix tables.
\item [\texttt{\textbackslash{}listoftables}]This works the same as \texttt{\textbackslash{}listoffigures}
above. Use \texttt{\textbackslash{}listofappendixtables} to list appendix
tables.
\end{lyxlist}

\subsection{Line Spacing Options}

\beavtex uses the default \LaTeX{} linespacing. The commands are
\texttt{\textbackslash{}singlespacing} (this is the default), \texttt{\textbackslash{}onehalfspacing}
and \texttt{\textbackslash{}doublespacing}. In order to set your thesis
in doublespace type, issue the \texttt{\textbackslash{}doublespacing}
command in the preamble.

When you reset the spacing, two important things will be reset. Your
Table of Contents and your Bibliography (and index, etc. if you include
them). If you want these single spaced, put them in a \texttt{singlespacing}
environment as so

\begin{lyxcode}
\textbackslash{}begin\{singlespacing\}\textbackslash{}tableofcontents\{\}\textbackslash{}end\{singlespacing\}
\end{lyxcode}
Likewise, for your bibliography, you would use something like

\begin{lyxcode}
\textbackslash{}begin\{singlespacing\}

\textbackslash{}bibliographystyle\{jawra\}

\textbackslash{}clearpage\textbackslash{}addcontentsline\{toc\}\{chapter\}\{Bibliography\}

\textbackslash{}bibliography\{bibfile\}

\textbackslash{}end\{singlespacing\}
\end{lyxcode}
Keep in mind, that you can singlespace virtually anything you want
this way, by starting with \texttt{\textbackslash{}begin\{singlespacing\}}
and ending with \texttt{\textbackslash{}end\{singlespacing\}}. For
instance:

\begin{singlespacing}

\begin{smallitem}
\item This is a list of items
\item with the smallitem environment.
\item and I want to make them print with single spacing
\end{smallitem}
\end{singlespacing}


\subsection{Controlling Page Alignment}

The optional pages (abstract, acknowlegements, contributors, dedication
and preface) as well as the body of the text can be aligned as you
see fit. The default is a justified body and left-aligned optional
pages, but some users may wish to change these settings. \emph{All
page alignment settings go into the preamble.}


\subsubsection{Body Alignment}

\begin{lyxlist}{00.00.0000}
\item [\texttt{\textbackslash{}pagealignment\{ALIGN\}}]Will reset the alignment
of the \emph{thesis body only}. \texttt{ALIGN} is one of \texttt{\textbackslash{}raggedright},
\texttt{\textbackslash{}raggedleft}, or \texttt{\textbackslash{}centering}.
It is unlikely that you would use them; however, if either the graduate
school or your department balks at a properly formatted text body,
just use

\begin{lyxcode}
\textbackslash{}pagealignment\{\textbackslash{}raggedright\}
\end{lyxcode}
and the result will be the older, ugly, technically incorrect, typewriter-style
text alignment.

\end{lyxlist}

\subsubsection{Optional Page Alignment}

The default alignment for the optional pages is \texttt{\textbackslash{}raggedright}.
Each of the four pages has a different suite of settings that can
be controlled, allowing you to have, for instance, a justified preface
and a centered dedications page. The commands are given below, and
are the exact same for each page. Replace \texttt{XXX} with either
\texttt{ack}, \texttt{abs}, \texttt{cont}, \texttt{ded} or \texttt{pref}
to use the command on the abstract, acknowledgements, contributors
dedications or preface pages, respectively.

\begin{lyxlist}{00.00.0000}
\item [\texttt{\textbackslash{}XXXpage{[}start|end{]}}]Use this command
and its arguments will be run immediately before and after the text
(not the title!) of the page. Be careful with these commands because
some commands will not end when the page ends. Thus, if you \emph{only}
use

\begin{lyxcode}
\textbackslash{}ackpagestart\{\textbackslash{}scshape\}
\end{lyxcode}
then all of the text from the Acknowlegements on--- throughout your
entire thesis--- will be in \noun{small caps}. Why? Because you
didn't end the command with something like 

\begin{lyxcode}
\textbackslash{}ackpageend\{\textbackslash{}upshape\}
\end{lyxcode}
\texttt{\textbackslash{}XXXpagestart} and \texttt{\textbackslash{}XXXpageend}
are fairly robust, but you should still be careful using them.

\item [\texttt{\textbackslash{}XXXpageclear}]This command clears the settings
for the page. \emph{Using this command by itself will result in justified
text.} By default, everypage has the same settings, which are

\begin{lyxcode}
\textbackslash{}XXXpagestart\{\textbackslash{}begin\{flushleft\}\}

\textbackslash{}XXXpageend\{\textbackslash{}end\{flushleft\}\}
\end{lyxcode}
Using \texttt{\textbackslash{}XXXpagestart} by itself will therefore
give you an error (because the page will still end with \texttt{\textbackslash{}end\{flushleft\}}.
If you always remember to issue a \texttt{\textbackslash{}XXXpageclear}
before changing any commands, and you'll be fine. Optionally, you
can use

\begin{lyxcode}
\textbackslash{}XXXpageend\{\}
\end{lyxcode}
which serves to clear out the ending command as well.

\item [\texttt{\textbackslash{}XXXpagecenter}]This command results in the
page's text being centered.
\end{lyxlist}

\paragraph{Page-Alignment Recap}

Here's a recap of the alignment options for the optional pages:

\begin{smallitem}
\item Use nothing and the page will be left-aligned. This is the default.
\item Use \texttt{\textbackslash{}XXXpageclear} and the page will be justified. 
\item Use \texttt{\textbackslash{}XXXpagecenter} and the page will be centered.
\item Use \texttt{\textbackslash{}XXXpagestart} and \texttt{\textbackslash{}XXXpageend}
to do something crazy.
\end{smallitem}
As an example, for this documentation, I used the following commands

\begin{lyxcode}
\textbackslash{}ackpageclear

\textbackslash{}abspageclear

\textbackslash{}dedpagecenter
\end{lyxcode}

\section{\beavtexnospace 's Idiosyncracies and Problems}

With any program, various choices have to be made by the programmer.
I've tried to limit these choices where I can. Still, there are a
few, and I'll add options for them if I know that people want them
changed.


\subsection{Choices}

At some points, I've no doubt made choices without creating options
so that the user can use to modify them. As people use this and complain
about those choices, I will either create an option or note them here.
Eventually, I may make another chapter which details the code so that
people can modify it. 


\subsection{Problems}

This is mainly a list of things I need to fix. If there is nothing
in the list, then there is nothing I know about.

\begin{comment}
\bibliographystyle{jawra}
\bibliography{bibfile}

\end{comment}

